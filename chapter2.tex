%
% Modified by Megan Patnott
% Last Change: Jan 18, 2013
%
%%%%%%%%%%%%%%%%%%%%%%%%%%%%%%%%%%%%%%%%%%%%%%%%%%%%%%%%%%%%%%%%%%%%%%%%
%
% Modified by Bryce Frentz
% Last Change: 2018
%
%%%%%%%%%%%%%%%%%%%%%%%%%%%%%%%%%%%%%%%%%%%%%%%%%%%%%%%%%%%%%%%%%%%%%%%%
%
% Sample Notre Dame Thesis/Dissertation
% Using Donald Peterson's ndthesis classfile
%
% Written by Jeff Squyres and Don Peterson
%
% Provided by the Information Technology Committee of
%   the Graduate Student Union
%   http://www.gsu.nd.edu/
%
% Nothing in this document is serious except the format.  :-)
%
% If you have any suggestions, comments, questions, please send e-mail
% to: ndthesis@gsu.nd.edu
%
%%%%%%%%%%%%%%%%%%%%%%%%%%%%%%%%%%%%%%%%%%%%%%%%%%%%%%%%%%%%%%%%%%%%%%%%

%
% Chapter 2
%

\chapter{Experimental setup and procedures}
\label{chap: experiment}

\section{Introduction}

This set of data was taken over the course of five* separate experiments. The first occurred at the University of Notre Dame's Nuclear Science Laboratory (NSL) in January of 2018 and covered the proton energy range of E$_{p}$ = 800 - 1200 keV. The experiment was then continued at the Compact Accelerator System for Performing Nuclear Astrophysics (CASPAR) facility at the Sanford Underground Research Facility located in Lead, South Dakota in three increments: February 2018, May 2018, and August / September 2018. These measurements covered the energy range from E$_{p}$ = 270 - 1200 keV, in order to measure the $^{14}$N$\left( p,\gamma \right) ^{15}$O reaction cross-section to compare the performance of the CASPAR facility to an above-ground laboratory. Finally, in MONTH TIME DATE THING, the final experiment was completed at the NSL, focusing on obtaining the lifetime of the 6793 keV state in $^{15}$O.

\section{Accelerators, beamline, and high-purity Germanium detectors}
\label{sec: accelerators, beamline, and detectors}



\subsection{Van de Graaff accelerators}

\subsection{Ion sources}

\subsection{Beamline}



\section{High-purity Germanium detectors}
\label{sec: HPGe detectors}


\section{Cross-section measurements}
\label{sec: nsl experiment}


\section{Lifetime measurement}
\label{sec: caspar experiment}




% % uncomment the following lines,
% if using chapter-wise bibliography
%
% \bibliographystyle{ndnatbib}
% \bibliography{example}
